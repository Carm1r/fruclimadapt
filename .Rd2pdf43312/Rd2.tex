\nonstopmode{}
\documentclass[a4paper]{book}
\usepackage[times,inconsolata,hyper]{Rd}
\usepackage{makeidx}
\usepackage[utf8]{inputenc} % @SET ENCODING@
% \usepackage{graphicx} % @USE GRAPHICX@
\makeindex{}
\begin{document}
\chapter*{}
\begin{center}
{\textbf{\huge Package `fruclimadapt'}}
\par\bigskip{\large \today}
\end{center}
\inputencoding{utf8}
\ifthenelse{\boolean{Rd@use@hyper}}{\hypersetup{pdftitle = {fruclimadapt: Evaluation Tools for Assessing Climate Adaptation of Fruit Tree Species}}}{}
\begin{description}
\raggedright{}
\item[Type]\AsIs{Package}
\item[Title]\AsIs{Evaluation Tools for Assessing Climate Adaptation of Fruit Tree Species}
\item[Version]\AsIs{0.4.5}
\item[Author]\AsIs{Carlos Miranda}
\item[Maintainer]\AsIs{Carlos Miranda }\email{carlos.miranda@unavarra.es}\AsIs{}
\item[Description]\AsIs{Climate is a critical component limiting growing range of plant species, which
also determines cultivar adaptation to a region. The evaluation of climate influence on
fruit production is critical for decision-making in the design stage of orchards and 
vineyards and in the evaluation of the potential consequences of future climate. Bio-
climatic indices and plant phenology are commonly used to describe the suitability of 
climate for growing quality fruit and to provide temporal and spatial information about 
regarding ongoing and future changes. 'fruclimadapt' streamlines the assessment of 
climate adaptation and the identification of potential risks for grapevines and fruit 
trees. Procedures in the package allow to i) downscale daily meteorological variables
to hourly values (Forster et al (2016) <}\Rhref{https://doi.org/10.5194/gmd-9-2315-2016}{doi:10.5194/gmd-9-2315-2016}\AsIs{>),
ii) estimate chilling and forcing heat accumulation (Miranda et al (2019)
<}\url{https://ec.europa.eu/eip/agriculture/sites/default/files/fg30_mp5_phenology_critical_temperatures.pdf}\AsIs{>),
iii) estimate plant phenology (Schwartz (2012) <}\Rhref{https://doi.org/10.1007/978-94-007-6925-0}{doi:10.1007/978-94-007-6925-0}\AsIs{>), iv) 
calculate bioclimatic indices to evaluate fruit tree and grapevine adaptation (e.g. Badr 
et al (2017) <}\Rhref{https://doi.org/10.3354/cr01532}{doi:10.3354/cr01532}\AsIs{>), v) estimate the incidence of weather-related disorders 
in fruits (e.g. Snyder and de Melo-Abreu (2005, ISBN:92-5-105328-6) and vi)
estimate plant water requirements (Allen et al (1998, ISBN:92-5-104219-5)). }
\item[Depends]\AsIs{R (>= 3.5.0)}
\item[Imports]\AsIs{data.table, magrittr, dplyr, zoo, lubridate}
\item[License]\AsIs{GPL (>= 3)}
\item[Encoding]\AsIs{UTF-8}
\item[LazyData]\AsIs{true}
\item[RoxygenNote]\AsIs{7.2.3}
\end{description}
\Rdcontents{\R{} topics documented:}
\inputencoding{utf8}
\HeaderA{Bigtop\_reqs}{Chill and heat requirements for Big Top nectarine}{Bigtop.Rul.reqs}
%
\begin{Description}\relax
Chill and forcing heat requirements for the phenological stages between
'bud swelling' (B, 51 in Baggliolini and BBCH scales, respectively) and 'ovary 
surrounded by dying sepal crown' (I, 72) in Big Top nectarine. For use in 
combination with the example dataset Tudela\_DW.
\end{Description}
%
\begin{Usage}
\begin{verbatim}
data("Bigtop_reqs")
\end{verbatim}
\end{Usage}
%
\begin{Format}
A data frame with 7 observations on the following 2 variables.
\begin{description}

\item[\code{Creq}] a numeric vector, chill portions
\item[\code{Freq}] a numeric vector, forcing heat as growing degree hours

\end{description}

\end{Format}
%
\begin{Details}\relax
Chill requirements are in chill portions, starting chill accummulation on 1 november
(day of year 305), forcing heat is as growing degree hours calculated with linear 
method with a base temperature of 4.7 C and no optimal and upper critical thresholds.
\end{Details}
%
\begin{Source}\relax
Miranda C, Santesteban LG and Royo JB. 2013. Evaluation and fitting of models for
determining peach phenological stages at a regional scale. Agric Forest Meteorol 
178-179: 129-139.
\end{Source}
\inputencoding{utf8}
\HeaderA{bioclim\_hydrotherm}{Calculation of hydrothermal viticultural indices (Branas, Dryness Index)}{bioclim.Rul.hydrotherm}
%
\begin{Description}\relax
This function calculates the hydrotermic index of Branas, Bernon and 
Levandoux (BBLI, Branas et al 1946) and the Dryness index (Riou et al 1994).
\end{Description}
%
\begin{Usage}
\begin{verbatim}
bioclim_hydrotherm(climdata, lat, elev)
\end{verbatim}
\end{Usage}
%
\begin{Arguments}
\begin{ldescription}
\item[\code{climdata}] a dataframe with daily weather data, including temperature
Required columns are Year, Month, Day, Tmax, Tmin and Prec. Optional columns 
are RHmax, RHmin, Rad and u2med.

\item[\code{lat}] the latitude of the site, in decimal degrees. Use positive values
for Northern latitudes and negatives for Southern.

\item[\code{elev}] the elevation of the site, in meters above sea level.
\end{ldescription}
\end{Arguments}
%
\begin{Details}\relax
The BBLI takes into account  the influence  of  both  
temperature  and  precipitation  on  grape yield  and  wine  quality.  
This  index  is  the  sum of  the products of monthly mean temperature 
(Tmean, in Celsius) and monthly  accumulated  precipitation  amount  (Prec,in mm)
during the April to September season (Northern Hemisphere) or October to 
February (Southern Hemisphere).

The  Dryness  index  (DI)  is  measured  based  on  an adaptation  of  
the  potential  water  balance  of  the  soil index of Riou (Riou et al., 
1994), developed specially for vineyard use. It  enables  the  characterization
of the  water  component  of  the  climate  in  a  grape-growing  region,
taking into account the climatic demand of a standard vineyard, evaporation
from bare soil, rainfall without deduction for surface runoff or drainage. 
It indicates the potential  water  availability  in  the  soil,  related  
to the level of dryness in a region (Tonietto and Carbonneau, 2004). The
index uses potential evapotranspiration calculated here with 
the Penman Monteith method.

Minimum data requirements to calculate the indices are daily temperatures 
(maximum and minimum temperatures, Tmax and Tmin) and rainfall (l m-2), 
whereas relative humidity (RHmax and RHmin, \%), solar radiation 
(Rad, MJ m-2 day-1) and mean wind speed at 2m height (u2med,m s-1) are optional. 
If missing, the function integrates FAO56 (Allen et al 1998) estimations 
for solar radiation and vapor pressure (air humidity) from daily temperatures. 
If there is no information available on wind speed, the function assumes a 
constant value of 2 m s-1.
\end{Details}
%
\begin{Value}
dataframe with the values of the indices for each season in the
climdata file.
\end{Value}
%
\begin{Author}\relax
Carlos Miranda, \email{carlos.miranda@unavarra.es}
\end{Author}
%
\begin{References}\relax
Allen RG, Pereira LS, Raes D, Smith M. 1998. Crop evapotranspiration. Guidelines
for computing crop water requirements. FAO Irrigation and drainage paper 56. Food 
and Agriculture Organization of the United Nations
Branas J, Bernon G, Levadoux L. 1946. Elements de Viticulture Generale. 
Imp. Dehan, Bordeaux
Riou C, Carbonneau A, Becker N, Caló A, Costacurta A, Castro R, Pinto PA, 
Carneiro LC, Lopes C, Climaco P, Panagiotou MM, Sotes V,Beaumond HC, Burril A, 
Maes J, Vossen P. 1994. Le determinisme climatique de la maturation du raisin: 
application au zonage de la teneur em sucre dans la communaute europenne. 
Office des Publications Officielles des Communautes Europennes: Luxembourg, 322pp.

Tonietto J, Carbonneau A. 2004. A multicriteria climatic classification system 
for grape-growing regions worldwide. Agricultural and Forest Meteorology, 124:81-97.
\end{References}
%
\begin{Examples}
\begin{ExampleCode}

# Select the appropiate columns from a larger dataset with date information
# in Year, Month, Day format, define the values for the parameters latitude 
# and elevation and estimate the hydrotermal indices on each year in the series.
library(magrittr)
library(dplyr)
Weather <- Tudela_DW %>%
   select(Year, Month, Day, Tmax, Tmin, Prec, RHmax, RHmin, Rad, u2med)
elevation <- 314
latitude <- 42.13132
Tudela_BHI <- bioclim_hydrotherm(Weather, latitude, elevation)

\end{ExampleCode}
\end{Examples}
\inputencoding{utf8}
\HeaderA{bioclim\_thermal}{Calculation of bioclimatic viticultural indices focusing on temperature}{bioclim.Rul.thermal}
%
\begin{Description}\relax
This function calculates the Growing Season Average Temperature (GST), the
Heliothermal Index (HI) of Huglin, the Winkler (WI) index, the Biologically
Effective Degree Day (BEDD) index and the Cool Night (CI) index.
\end{Description}
%
\begin{Usage}
\begin{verbatim}
bioclim_thermal(climdata, lat)
\end{verbatim}
\end{Usage}
%
\begin{Arguments}
\begin{ldescription}
\item[\code{climdata}] a dataframe with daily maximum and minimum temperatures.
Must contain the columns Year, Month, Day, Tmax, Tmin.

\item[\code{lat}] the latitude of the site, in decimal degrees. Use positive values
for Northern latitudes and negatives for Southern.
\end{ldescription}
\end{Arguments}
%
\begin{Details}\relax
GST index correlates broadly to the maturity potential for grape cultivars 
grown across many wine regions and provides the basis for zoning viticultural
areas in both hemispheres (Hall and Jones, 2009). It is calculated by taking
the average of the growing season (April-October in Northern hemisphere, October
-April in Southern hemisphere).

HI (Huglin, 1978) is a bioclimatic heat index for viticulture regions using 
heliothermic potential, which calculates the temperature sum above 10ºC from 
April until September (Northern hemisphere) or from October until March (Southern
hem.). The index takes into consideration daily maximum and average temperature, 
and slightly modifies the calculated total using the latitude of the location.

WI index (Amerine and Winkler, 1944), also known as growing degree days (GDD) 
classifies regions based on the accumulation of heat summation units by adding up 
hours above 10ºC during the growing season.

BEDD index (Gladstones, 1992) is another variant on calculating heat summation 
which incorporates upper and lower temperature thresholds (accounts for heat 
accumulation between 10 and 19ºC) and a day length correction similar to HI.

CI index (Tonietto, 1999) takes into account the minimum temperature during 
grape maturation, which is normally the average minimum air temperature in 
September/March (Northern or Southern hemispheres, respectively).
\end{Details}
%
\begin{Value}
data frame with the values of the indices. It contains the columns
Year, CI, GST, BEDD, HI, WI
\end{Value}
%
\begin{Author}\relax
Carlos Miranda, \email{carlos.miranda@unavarra.es}
\end{Author}
%
\begin{References}\relax
Amerine MA and Winkler AJ. 1944. Composition and quality of musts and wines
of California grapes. Hilgardia 15: 493-675.

Gladstones J. 1992. Viticulture and environment. Winetitles, Adelaide, Australia 

Hall A., Jones GV. 2009. Effect of potential atmospheric warming on 
temperature-based indices describing Australian winegrape growing conditions.
Aust J Grape Wine Res 15. 97-119.

Huglin P. 1978. Noveau mode d'evaluation des possibilites héliothermiques
d'un milieu viticole. In: Proceedings of the Symposium International sur
l'ecologie de la Vigne. Ministére de l'Agriculture et de l'Industrie 
Alimentaire, Contança pp 89-98.

Tonietto J. 1999. Les macroclimats viticoles mondiaux et l'influence du
mésoclimat sur la typicité de la Syrah et du Muscat de Hambourg dans le
sud de la France: methodologie de carácterisation. Thése Doctorat. Ecole 
Nationale Supérieure Agronomique, Montpellier, 233pp.
\end{References}
%
\begin{Examples}
\begin{ExampleCode}

# Select the appropiate columns from a larger dataset with date information
# in Year, Month, Day format, and estimate indices on each year in the series.
library(magrittr)
library(dplyr)
Weather <- Tudela_DW %>%
   select(Year, Month, Day, Tmax, Tmin)
latitude <- 42.13132
Tudela_BTI <- bioclim_thermal(Weather, latitude)

\end{ExampleCode}
\end{Examples}
\inputencoding{utf8}
\HeaderA{chill\_hours}{Calculation of chill hours from hourly temperature data (Weinberger model)}{chill.Rul.hours}
%
\begin{Description}\relax
The function calculates chill hours using the Weinberger (1950),
or 0-7.2ºC method. Sums chill hours over winter, with one chill
hour accumulated for hourly temperatures between 0 and 7.2°C.
This is a classic method but highly inefficient, particularly
for warm regions and in climate change scenarios, as it disregards
temperature ranges that are now known to contribute to the fulfilment
of chilling requirements. For that reason, its use is not recommended,
it is offered only for educational purposes (i.e. comparison of model
performance) and compatibility with older bibliography.
\end{Description}
%
\begin{Usage}
\begin{verbatim}
chill_hours(climdata, Start)
\end{verbatim}
\end{Usage}
%
\begin{Arguments}
\begin{ldescription}
\item[\code{climdata}] a dataframe with hourly temperature data. It
must contain the columns Year, Month, Day, DOY, Hour, Temp.

\item[\code{Start}] parameter indicating the day of the year when chill
accumulation is supposed to start.
\end{ldescription}
\end{Arguments}
%
\begin{Value}
dataframe with the chill accumulated for all the seasons in the
dataset. Seasons begin at the start date and end the day before the start
date of the following year. It contains the columns Year, Month, Day, 
DOY and Chill.
\end{Value}
%
\begin{Author}\relax
Carlos Miranda, \email{carlos.miranda@unavarra.es}
\end{Author}
%
\begin{References}\relax
Weinberger JH, 1950. Chilling requirements of peach varieties. Proc Am Soc
Hortic Sci 56, 122-128.
\end{References}
%
\begin{Examples}
\begin{ExampleCode}

# Generate hourly temperatures
library(magrittr)
library(dplyr)
library(lubridate)
Tudela_HT <- hourly_temps(Tudela_DW,42.13132)
# Calculate chill as chill hours, starting on DOY 305
Chill_h <- chill_hours(Tudela_HT,305)

\end{ExampleCode}
\end{Examples}
\inputencoding{utf8}
\HeaderA{chill\_portions}{Calculation of chill portions from hourly temperature data (Dynamic model)}{chill.Rul.portions}
%
\begin{Description}\relax
The function calculates chill portions according to the Dynamic model
proposed by Fishman et al. (1987a,b), using the formulas extracted by
Luedeling et al (2009) from functions produced by Erez and Fishman (1990), 
available at the University of California, Agriculture and Natural Resources
(UC ANR) website http://ucanr.edu/sites/fruittree/files/49319.xls. To date, 
chill portions is the best existing model for most growing regions, so chill
fulfilment should  be calculated preferably using this method, especially
when transferring varieties from one region to another, or in studies on 
climate change.
\end{Description}
%
\begin{Usage}
\begin{verbatim}
chill_portions(climdata, Start)
\end{verbatim}
\end{Usage}
%
\begin{Arguments}
\begin{ldescription}
\item[\code{climdata}] a dataframe with hourly temperature data. It
must contain the columns Year, Month, Day, DOY, Temp.

\item[\code{Start}] parameter indicating the day of the year when chill
accumulation is supposed to start.
\end{ldescription}
\end{Arguments}
%
\begin{Value}
dataframe with the chill accumulated for all the seasons in the
dataset. Seasons begin at the start date and end the day before the start
date of the following year.
It contains the columns Year, Month, Day, DOY, Chill
\end{Value}
%
\begin{Author}\relax
Carlos Miranda, \email{carlos.miranda@unavarra.es}
\end{Author}
%
\begin{References}\relax
Erez A, Fishman S, Linsley-Noakes GC and Allan P, 1990. The dynamic model for
rest completion in peach buds. Acta Horticulturae 276, 165-174.

Fishman S, Erez A and Couvillon GA, 1987a. The temperature dependence of
dormancy breaking in plants - computer simulation of processes studied under
controlled temperatures. Journal of Theoretical Biology 126, 309-321.

Fishman S, Erez A and Couvillon GA, 1987b. The temperature dependence of
dormancy breaking in plants - mathematical analysis of a two-step model
involving a cooperative transition. Journal of Theoretical Biology 124,
473-483.

Luedeling E, Zhang M, Luedeling V and Girvetz EH, 2009. Sensitivity of
winter chill models for fruit and nut trees to climatic changes expected in
California's Central Valley. Agriculture, Ecosystems and Environment 133,
23-31.
\end{References}
%
\begin{Examples}
\begin{ExampleCode}

# Generate hourly temperatures
library(magrittr)
library(dplyr)
library(lubridate)
Weather <- Tudela_DW %>%
   filter (Tudela_DW$Year<=2003)
Tudela_HT <- hourly_temps(Tudela_DW,42.13132)
# Calculate chill as chill portions, starting on DOY 305
Chill_p <- chill_portions(Tudela_HT,305)

\end{ExampleCode}
\end{Examples}
\inputencoding{utf8}
\HeaderA{chill\_units}{Calculation of chill units from hourly temperature data (Utah model)}{chill.Rul.units}
%
\begin{Description}\relax
The function calculates chill units using the Utah model (Richardson
et al, 1974). This model is characterized by differential weighting of
temperature ranges, including negative weights for temperatures
above 15.9°C. This model recognizes that different temperatures
vary in effectiveness in accumulating chill as well as a negative
influence of high temperatures on previously accumulated chill.
Chill Units (Utah or Anderson model) perform better than chill hours
for a wider range of climates, and it could be considered as the
'reference' method nowadays, but it is ill-suited for warm or
Mediterranean conditions. To date, Chill portions is the best
existing model for most growing regions, so chill fulfilment
should preferably be calculated using that method, especially
when transferring varieties from one region to another, or in
studies on climate change.
\end{Description}
%
\begin{Usage}
\begin{verbatim}
chill_units(climdata, Start)
\end{verbatim}
\end{Usage}
%
\begin{Arguments}
\begin{ldescription}
\item[\code{climdata}] a dataframe with hourly temperature data. It
must contain the columns Year, Month, Day, DOY, Temp.

\item[\code{Start}] parameter indicating the day of the year when chill
accumulation is supposed to start.
\end{ldescription}
\end{Arguments}
%
\begin{Value}
dataframe with the chill accumulated for all the seasons in the
dataset. Seasons begin at the start date and end the day before the start
date of the following year.
It contains the columns Year, Month, Day, DOY, Chill
\end{Value}
%
\begin{Author}\relax
Carlos Miranda, \email{carlos.miranda@unavarra.es}
\end{Author}
%
\begin{References}\relax
Richardson EA, Seeley SD and Walker DR, 1974. A model for estimating the
completion of rest for Redhaven and Elberta peach trees. HortScience 9,
331-332.
\end{References}
%
\begin{Examples}
\begin{ExampleCode}

# Generate hourly temperatures
library(magrittr)
library(dplyr)
library(lubridate)
Weather <- Tudela_DW %>%
   filter (Tudela_DW$Year<=2003)
Tudela_HT <- hourly_temps(Tudela_DW,42.13132)
# Calculate chill as chill units, starting on DOY 305
Chill_u <- chill_units(Tudela_HT,305)

\end{ExampleCode}
\end{Examples}
\inputencoding{utf8}
\HeaderA{color\_potential}{Evaluation of weather conditions for anthocyanin formation in apple skin}{color.Rul.potential}
%
\begin{Description}\relax
This function estimates the number of days that can be 
considered as highly favorable or unfavorable for anthocyanin 
accumulation in the skin of red apple cultivars during a
user defined pre-harvest period (30 days  by default). 
A highly favorable day (Cool day) is considered when the daily 
maximum temperature is below 26ºC, a highly unfavorable day (Hot day)
when the minimum temperature is above 20ºC (Lin-Wang et al, 2011). It
also calculates an empirical index in which daily thermal amplitude 
is corrected to account for the effective range of temperatures for
anthocyanin accumulation in the skin (TA\_cef). The index considers that
daily temperatures above 26ºC are increasingly less favorable for 
anthocyanin formation, and thus calculates a corrected maximum 
temperature using a linear function up to 35ºC, where it is left 
constant at a value of 16, so that the adjusted daily thermal amplitude 
for Tmax>26ºC is smaller than the observed. 
The average of maximum and minimum temperatures during the same 
period is also provided. The function allows testing for several 
harvest dates.
\end{Description}
%
\begin{Usage}
\begin{verbatim}
color_potential(climdata, harvest, span = 30)
\end{verbatim}
\end{Usage}
%
\begin{Arguments}
\begin{ldescription}
\item[\code{climdata}] a dataframe with daily maximum and minimum temperatures.
Must contain the columns Year, Month, Day, Tmax, Tmin.

\item[\code{harvest}] a vector with expected harvest days
(expressed as day of the year)

\item[\code{span}] the period (in days) before harvest that will be analyzed. By 
default, this parameter is set in 30 days.
\end{ldescription}
\end{Arguments}
%
\begin{Value}
dataframe with the number of highly favorable (Cool\_d) 
and unfavorable (Hot\_d) days for apple red color, as well as the sums of
the observed (TA\_obs) and effective (TA\_cef) daily thermal amplitudes.
The average of the maximum (Tmax\_avg) and minimum (Tmin\_avg) 
temperatures for each year (Year) in the series during the 
30 days previous to each harvest date (Day\_h) is also provided.
\end{Value}
%
\begin{Author}\relax
Carlos Miranda, \email{carlos.miranda@unavarra.es}
\end{Author}
%
\begin{References}\relax
Lin-Wang K, Micheletti D et al, 2011. High temperature reduces apple fruit
colour via modulation of the anthocyanin regulatory complex. Plant, Cell
and Environment 34, 1176-1190.
\end{References}
%
\begin{Examples}
\begin{ExampleCode}

# Select the appropiate columns from Tudela_DW example dataset, create
# a vector or harvest dates and estimate the number favorable and 
# unfavorable days on each year in the dataset.
library(magrittr)
library(dplyr)
Weather <- Tudela_DW %>%
   select(Year, Month, Day, Tmax, Tmin) %>% 
   filter (Tudela_DW$Year<=2002)
harvest <- c(225, 250, 275)
Color_assess <- color_potential(Weather, harvest)

\end{ExampleCode}
\end{Examples}
\inputencoding{utf8}
\HeaderA{coolness\_index}{Calculation of night coolness index}{coolness.Rul.index}
%
\begin{Description}\relax
This function calculates a night coolness index based in the
Cool Night index of Tonietto (1999). Instead of calculating 
the mean of minimum temperatures in September/March (Northern
or Southern hemispheres, respectively), this function allows 
to define the harvest date and the number of days that will be
analyzed (by default, 30 days), and calculates the mean of minimum
temperatures in the in the specified period of days before harvest.
The function allows testing for several harvest dates simultaneously.
\end{Description}
%
\begin{Usage}
\begin{verbatim}
coolness_index(climdata, harvest, span = 30)
\end{verbatim}
\end{Usage}
%
\begin{Arguments}
\begin{ldescription}
\item[\code{climdata}] a dataframe with daily maximum and minimum temperatures.
Must contain the columns Year, Month, Day, Tmax, Tmin.

\item[\code{harvest}] a vector with expected harvest days
(expressed as day of the year)

\item[\code{span}] the period (in days) before harvest that will be analyzed. By 
default, this parameter is set in 30 days.
\end{ldescription}
\end{Arguments}
%
\begin{Value}
dataframe with the values of the indices. It contains the
columns Year, Harvest, Coolness
\end{Value}
%
\begin{Author}\relax
Carlos Miranda, \email{carlos.miranda@unavarra.es}
\end{Author}
%
\begin{References}\relax
Tonietto J. 1999. Les macroclimats viticoles mondiaux et l'influence du
mésoclimat sur la typicité de la Syrah et du Muscat de Hambourg dans le
sud de la France: methodologie de carácterisation. Thése Doctorat. Ecole 
Nationale Supérieure Agronomique, Montpellier, 233pp.
\end{References}
%
\begin{Examples}
\begin{ExampleCode}

# Select the appropiate columns from the Tudela_DW example dataset,
# create a vector or harvest dates and estimate the coolness index 
# for the 30 days prior to harvest on each year in the dataset.
library(magrittr)
library(dplyr)
Weather <- Tudela_DW %>%
   select(Year, Month, Day, Tmax, Tmin) %>% 
   filter (Tudela_DW$Year<=2002)
harvest <- c(225, 250, 275)
coolness <- coolness_index(Weather, harvest)

\end{ExampleCode}
\end{Examples}
\inputencoding{utf8}
\HeaderA{Dates\_BT}{Example phenological dates for Big Top nectarine in Tudela}{Dates.Rul.BT}
%
\begin{Description}\relax
Estimated dates (as day of the year, DOY) for bloom, early fruit growth stages 
and harvest for Big Top Nectarine in Tudela (2001-2010). Dates have been 
estimated using the functions included in this package.
\end{Description}
%
\begin{Usage}
\begin{verbatim}
data("Dates_BT")
\end{verbatim}
\end{Usage}
%
\begin{Format}
A data frame with 10 observations on the following 6 variables.
\begin{description}

\item[\code{Year}] a numeric vector, the year of the observation
\item[\code{sbloom}] a numeric vector, beggining of bloom as DOY
\item[\code{ebloom}] a numeric vector, end of bloom as DOY
\item[\code{Start\_ing}] a numeric vector, beggining of early growth stage as DOY
\item[\code{End\_ing}] a numeric vector, end of early growth stage as DOY
\item[\code{Harvest}] a numeric vector, harvest date as DOY

\end{description}

\end{Format}
\inputencoding{utf8}
\HeaderA{DTR}{Calculation of the diurnal temperature range (DTR)}{DTR}
%
\begin{Description}\relax
This function calculates the mean diurnal temperature range (DTR) for
a custom period. Mean DTR is obtained by subtracting the daily minimum
temperature (Tmin) from daily maximum temperature (Tmax) and then
averaged for the period defined by the user, provided as the initial
(init) and end (end) date expressed as days of the year. The function
requires the initial and end dates to be in the same year.
\end{Description}
%
\begin{Usage}
\begin{verbatim}
DTR(climdata, init_d, end_d)
\end{verbatim}
\end{Usage}
%
\begin{Arguments}
\begin{ldescription}
\item[\code{climdata}] a dataframe with daily maximum and minimum temperatures.
Must contain the columns Year, Month, Day, Tmax, Tmin.

\item[\code{init\_d}] the initial date (as day of the year) of the evaluation period

\item[\code{end\_d}] the end date (as day of the year) of the evaluation period
\end{ldescription}
\end{Arguments}
%
\begin{Value}
dataframe with the value of DTR for each year in the series. 
It contains the columns Year, First\_d, Last\_d, DTR
\end{Value}
%
\begin{Author}\relax
Carlos Miranda, \email{carlos.miranda@unavarra.es}
\end{Author}
%
\begin{Examples}
\begin{ExampleCode}

# Select the appropiate columns from the Tudela_DW example dataset,
# and estimate the mean DTR for July on each year in the dataset.
library(magrittr)
library(dplyr)
Weather <- Tudela_DW %>%
   select(Year, Month, Day, Tmax, Tmin)
DTR_July <- DTR(Weather, 182, 212)

\end{ExampleCode}
\end{Examples}
\inputencoding{utf8}
\HeaderA{ET\_penman}{Calculation of daily potential evapotranspiration by Penman (1948) method}{ET.Rul.penman}
%
\begin{Description}\relax
This function calculates the potential evapotranspiration (ETref) using daily 
weather data and the Penman (1948) method
\end{Description}
%
\begin{Usage}
\begin{verbatim}
ET_penman(climdata, lat, elev)
\end{verbatim}
\end{Usage}
%
\begin{Arguments}
\begin{ldescription}
\item[\code{climdata}] a dataframe with daily weather data.
Must contain the columns Year, Month, Day, Tmax, Tmin, RHmax, RHmin, Rad, u2med.

\item[\code{lat}] the latitude of the site, in decimal degrees.

\item[\code{elev}] the elevation of the site, in meters above sea level.
\end{ldescription}
\end{Arguments}
%
\begin{Details}\relax
This version of the function requires the user to supply in weather data daily
values for temperature (Tmax and Tmin), relative humidity (RHmax and RHmin), 
solar radiation (Rad in MJ m-2 day-1) and mean wind speed at 2m height 
(u2med in m s-1).
\end{Details}
%
\begin{Value}
dataframe where Date, DOY and ET columns have been added to the ones
in climadata data frame.
\end{Value}
%
\begin{Author}\relax
Carlos Miranda, \email{carlos.miranda@unavarra.es}
\end{Author}
%
\begin{References}\relax
Penman HL 1948.Natural evaporation from open water, bare soil and grass.
Proc. R. Soc. Lond. 193:120–145.
\end{References}
%
\begin{Examples}
\begin{ExampleCode}
 # Calculate ET by Penman method in the Tudela_DW example dataset
data(Tudela_DW)
library(magrittr)
library(dplyr)
elevation <- 314
latitude <- 42.13132
ET_Penman <- ET_penman(Tudela_DW, elevation, latitude) 

\end{ExampleCode}
\end{Examples}
\inputencoding{utf8}
\HeaderA{ET\_penman\_monteith}{Calculation of daily reference evapotranspiration by Penman-Monteith method}{ET.Rul.penman.Rul.monteith}
%
\begin{Description}\relax
This function calculates the reference evapotranspiration (ETref) for short
(ETos) and tall (ETrs) canopies using daily weather data. The method is based
on the FAO56 guidelines (Allen et al, 1998) and on the standardized Penman Monteith 
equation from the Environmental Water Resources Institute of the American 
Society of Civil Engineers (Allen et al, 2005).
\end{Description}
%
\begin{Usage}
\begin{verbatim}
ET_penman_monteith(climdata, lat, elev)
\end{verbatim}
\end{Usage}
%
\begin{Arguments}
\begin{ldescription}
\item[\code{climdata}] a dataframe with daily weather data.
Required columns are Year, Month, Day, Tmax and Tmin. Optional columns are
RHmax, RHmin, Rad and u2med.

\item[\code{lat}] the latitude of the site, in decimal degrees.

\item[\code{elev}] the elevation of the site, in meters above sea level.
\end{ldescription}
\end{Arguments}
%
\begin{Details}\relax
Minimum data requirements to calculate ET are daily temperatures (maximum 
and minimum temperatures, Tmax and Tmin), whereas relative humidity (RHmax and 
RHmin), solar radiation (Rad, MJ m-2 day-1) and mean wind speed at 2m height
(u2med,m s-1) are optional. If missing, the function integrates FAO56 estimations 
for solar radiation and vapor pressure (air humidity) from daily 
temperatures. If there is no information available on wind speed, the function 
assumes a constant value of 2 m s-1.
\end{Details}
%
\begin{Value}
dataframe with Year, Month, Day, DOY, ETos and ETrs values.
\end{Value}
%
\begin{Author}\relax
Carlos Miranda, \email{carlos.miranda@unavarra.es}
\end{Author}
%
\begin{References}\relax
Allen RG, Pereira LS, Raes D, Smith M. 1998. Crop evapotranspiration. Guidelines
for computing crop water requirements. FAO Irrigation and drainage paper 56. Food 
and Agriculture Organization of the United Nations

Allen RG, Walter IA, Elliott RL, Howell TA, Itenfisu D, Jensen ME, Snyder RL 2005. 
The ASCE standardized reference evapotranspiration equation. Reston, VA:American 
Society of Civil Engineers. 59 p.
\end{References}
%
\begin{Examples}
\begin{ExampleCode}

# Calculate ET by Penman-Monteith method in the Tudela_DW example dataset
library(magrittr)
library(dplyr)
elevation <- 314
latitude <- 42.13132
ET_PM <- ET_penman_monteith(Tudela_DW, latitude, elevation)

\end{ExampleCode}
\end{Examples}
\inputencoding{utf8}
\HeaderA{GDD\_linear}{Calculates growing degree days (GDD) using a linear method}{GDD.Rul.linear}
%
\begin{Description}\relax
The function calculates the daily heat unit accumulation (GDD)
from daily temperature data with a linear method based on averaging
the daily maximum and minimum temperatures (Arnold, 1960). GDD are
calculated by subtracting the plant's lower base temperature (Tb) 
from the average daily air temperature. The user can define an upper 
temperature threshold (Tu) so that all temperatures above Tu will 
have equal value in GDD summation.
\end{Description}
%
\begin{Usage}
\begin{verbatim}
GDD_linear(Temp_Day, Tb, Tu = 999)
\end{verbatim}
\end{Usage}
%
\begin{Arguments}
\begin{ldescription}
\item[\code{Temp\_Day}] a dataframe of daily temperatures. This data frame must
have a column for Year, Month, Day, and daily minimum (Tmin) and
maximum (Tmax) temperatures.

\item[\code{Tb}] the base temperature required to calculate GDD.

\item[\code{Tu}] an optional upper temperature threshold.
\end{ldescription}
\end{Arguments}
%
\begin{Value}
dataframe consisting of the columns Year, Month, Day, Tmax, Tmin,
Tmean and GDD.
\end{Value}
%
\begin{Author}\relax
Carlos Miranda, \email{carlos.miranda@unavarra.es}
\end{Author}
%
\begin{References}\relax
Arnold,  C.Y.  1960.  Maximum-minimum temperatures as a basis for computing
heat units. Proc.Amer. Soc. Hort. Sci. 76:682–692.
\end{References}
%
\begin{Examples}
\begin{ExampleCode}

# Calculate GDD in the example dataset using 4.5ºC as base temperature and no 
# upper threshold.
library(magrittr)
library(dplyr)
GDD <- GDD_linear(Tudela_DW,4.5)
# Calculate GDD in the example dataset using 4.5ºC as base temperature and an 
# upper threshold at 25ºC.
GDD <- GDD_linear(Tudela_DW,4.5,25)


\end{ExampleCode}
\end{Examples}
\inputencoding{utf8}
\HeaderA{GDH\_asymcur}{Calculates growing degree hours (GDH) using ASYMCUR method}{GDH.Rul.asymcur}
%
\begin{Description}\relax
The function calculates the daily heat unit accumulation (GDH)
from hourly temperature data, using the ASYMCUR model
proposed by Anderson et al (1986). The model is a refinement
of the linear model proposed by Anderson and Seeley (1992) defined
by a base, optimum and critical temperature. Heat accumulation
begins when temperatures are above a minimum (base temperature,
Tb), and growth increases with temperature up to a
point (optimum temperature, Topt) at which there is no longer
an increase. The critical temperature (Tcrit) is the temperature
above which growth ceases. The difference of ASYMCUR model with 
the linear by Anderson and Seeley (1992)is that the former uses 
an asymmetric curvilinear relationship to model GDH accumulation. 
The function allows the user to define Tb, Topt and Tcrit, and uses
as default the values set by Anderson et al (1986) for fruit trees:
Tb=4ºC, Topt=25ºC and Tcrit=36ºC.
\end{Description}
%
\begin{Usage}
\begin{verbatim}
GDH_asymcur(Hourdata, Tb = 4, Topt = 25, Tcrit = 36)
\end{verbatim}
\end{Usage}
%
\begin{Arguments}
\begin{ldescription}
\item[\code{Hourdata}] a dataframe of hourly temperatures. This data frame
must have a column for Year, Month, Day, DOY (day of year), Hour,
Temp (hourly temperature).

\item[\code{Tb}] the base temperatures to calculate GDH

\item[\code{Topt}] the optimal temperatures to calculate GDH

\item[\code{Tcrit}] the critical temperature
\end{ldescription}
\end{Arguments}
%
\begin{Value}
dataframe with daily data. It contains the columns Date,
Year, Month, Day, DOY (day of the year), and GDH
\end{Value}
%
\begin{Author}\relax
Carlos Miranda, \email{carlos.miranda@unavarra.es}
\end{Author}
%
\begin{References}\relax
Anderson JL, Richardson EA and Kesner CD, 1986. Validation of chill
unit and flower bud phenology models for 'Montmorency' sour cherry.
Acta Horticulturae 184, 71-75.
Anderson JL and Seeley SD, 1992. Modelling strategy in pomology:
Development of the Utah models. Acta Horticulturae 313, 297-306.
\end{References}
%
\begin{Examples}
\begin{ExampleCode}

# Generate hourly temperatures for the example dataset
library(magrittr)
library(dplyr)
library(lubridate)
Weather <- Tudela_DW %>%
   filter (Tudela_DW$Year==2003)
Tudela_HT <- hourly_temps(Weather,42.13132)
# Calculate GDH using default threshold temperatures
GDH_default <- GDH_asymcur(Tudela_HT)
# Calculate GDH using as custom set temperature thresholds
# Tb=4.5, Topt=22 and Tcrit=32
GDH_custom <- GDH_asymcur(Tudela_HT, 4.5, 22, 32)


\end{ExampleCode}
\end{Examples}
\inputencoding{utf8}
\HeaderA{GDH\_linear}{Calculates growing degree hours (GDH) using a linear method}{GDH.Rul.linear}
%
\begin{Description}\relax
The function calculates the daily heat unit accumulation (GDH)
from hourly temperature data, using a standard linear model or the
linear model proposed by Anderson and Seeley (1992). The standard
model is defined by a base temperature, and the Anderson and Seeley 
(1992) includes also optimum and critical temperatures. In both
variants, heat accumulation begins when temperatures are above a 
minimum (base temperature, Tb), and growth increases linearly with 
temperature. In the Anderson and Seeley (1992) variant, growth no
longer increases once the optimum temperature (Topt) is reached, 
meaning that GDH above it are constant. The critical temperature 
(Tcrit) is the temperature above which growth ceases (i.e. GDH=0). 
The function allows the user to define Tb, Topt and Tcrit, and uses
as default the values set by Anderson et al (1986) for fruit trees: 
Tb=4ºC, Topt=25ºC and Tcrit=36ºC. In the standard linear model with
upper thresholds, use Topt = 999 and Tcrit = 999.
\end{Description}
%
\begin{Usage}
\begin{verbatim}
GDH_linear(Hourdata, Tb = 4, Topt = 25, Tcrit = 36)
\end{verbatim}
\end{Usage}
%
\begin{Arguments}
\begin{ldescription}
\item[\code{Hourdata}] a dataframe of hourly temperatures. This data frame
must have a column for Year, Month, Day, DOY (day of year),Hour, and
Temp (hourly temperature).

\item[\code{Tb}] the base temperatures to calculate GDH

\item[\code{Topt}] an optional optimal temperatures to calculate GDH

\item[\code{Tcrit}] an optional critical temperature
\end{ldescription}
\end{Arguments}
%
\begin{Value}
dataframe with daily data. It contains the columns Date,
Year, Month, Day, DOY (day of the year), and GDH
\end{Value}
%
\begin{Author}\relax
Carlos Miranda, \email{carlos.miranda@unavarra.es}
\end{Author}
%
\begin{References}\relax
Anderson JL and Seeley SD, 1992. Modelling strategy in pomology:
Development of the Utah models. Acta Horticulturae 313, 297-306.
\end{References}
%
\begin{Examples}
\begin{ExampleCode}

# Generate hourly temperatures for the example dataset
library(magrittr)
library(dplyr)
library(lubridate)
Weather <- Tudela_DW %>%
   filter (Tudela_DW$Year==2003)
Tudela_HT <- hourly_temps(Weather,42.13132)
# Calculate GDH using default threshold temperatures
GDH_default <- GDH_linear(Tudela_HT)
# Calculate GDH using an optimal temperature threshold with 
# no critical threshold
GDH_custom <- GDH_linear(Tudela_HT, 4.5, 22, 999)

\end{ExampleCode}
\end{Examples}
\inputencoding{utf8}
\HeaderA{hourly\_RH}{Estimation of the hourly relative humidity on a daily series}{hourly.Rul.RH}
%
\begin{Description}\relax
This function estimates the hourly relative humidity (RH),
using daily temperature and humidity data. Hourly humidity is
estimated with the formula proposed by Waichler and Wigmosta (2003)
which require maximum and minimum values of daily temperature
and relative humidity.
\end{Description}
%
\begin{Usage}
\begin{verbatim}
hourly_RH(climdata, lat)
\end{verbatim}
\end{Usage}
%
\begin{Arguments}
\begin{ldescription}
\item[\code{climdata}] a dataframe with daily temperatures and RH for
each day in a series. Must contain the columns Year, Month, Day,
Tmax, Tmin, RHmax (daily maximum relative humidity) and RHmin
(minimum daily relative humidity).

\item[\code{lat}] the latitude of the site, in decimal degrees.
\end{ldescription}
\end{Arguments}
%
\begin{Value}
dataframe with columns Date, Year, Month, Day, DOY,
Hour, Temp and RH
\end{Value}
%
\begin{Author}\relax
Carlos Miranda, \email{carlos.miranda@unavarra.es}
\end{Author}
%
\begin{References}\relax
Waichler SR and Wigmosta MS, 2003. Development of hourly meteorological
values from daily data and significance to hydrological modeling at H.J.
Andrews experimental forest. Journal of Hydrometeorology 4, 251-263.
\end{References}
%
\begin{Examples}
\begin{ExampleCode}

# Generate hourly relative humidity
library(magrittr)
library(dplyr)
library(lubridate)
Weather <- Tudela_DW %>%
   filter (Tudela_DW$Year==2003)
Tudela_HRH <- hourly_RH(Weather, 42.13132)

\end{ExampleCode}
\end{Examples}
\inputencoding{utf8}
\HeaderA{hourly\_temps}{Estimation of hourly temperatures from daily data}{hourly.Rul.temps}
%
\begin{Description}\relax
This function generates hourly temperatures from daily maximum and
minimum values, using the method proposed by Linvill (1990), which also
requires sunset and sunrise calculation for each day in the series.
Sunset and sunrise hours for a location are internally estimated using 
the function solar\_times from the latitude and the day of the year (DOY)
using the equations by Spencer (1971) and Almorox et al. (2005).
\end{Description}
%
\begin{Usage}
\begin{verbatim}
hourly_temps(climdata, latitude)
\end{verbatim}
\end{Usage}
%
\begin{Arguments}
\begin{ldescription}
\item[\code{climdata}] a data frame containing the columns Year, Month, Day,
Tmax and Tmin. Data must not contain any gap.

\item[\code{latitude}] the latitude of the site, in decimal degrees.
\end{ldescription}
\end{Arguments}
%
\begin{Value}
a dataframe containing the columns Date, Year, Month, Day, DOY,
Hour, Sunrise (hour of sunrise), Sunset (hour of sunset), Daylength and
Temp (hourly temperature).
\end{Value}
%
\begin{Author}\relax
Carlos Miranda, \email{carlos.miranda@unavarra.es}
\end{Author}
%
\begin{References}\relax
Almorox J, Hontoria C and Benito M, 2005. Statistical validation of
daylength definitions for estimation of global solar radiation in Toledo,
Spain. Energy Conversion and Management 46, 1465-1471.

Linvill DE, 1990. Calculating chilling hours and chill units from daily
maximum and minimum temperature observations. HortScience 25, 14-16.

Spencer JW, 1971. Fourier series representation of the position of the Sun.
Search 2, 172.
\end{References}
%
\begin{Examples}
\begin{ExampleCode}

# Generate hourly temperatures
library(magrittr)
library(dplyr)
library(lubridate)
Tudela_HT <- hourly_temps(Tudela_DW,42.13132)

\end{ExampleCode}
\end{Examples}
\inputencoding{utf8}
\HeaderA{hourly\_windspeed}{Estimation of the hourly wind speed from daily mean data}{hourly.Rul.windspeed}
%
\begin{Description}\relax
This function estimates the hourly wind speed from a
dataset with mean daily wind speeds. Hourly wind speeds
from daily values are computed using the formulas proposed
by Guo et al (2016), using mean daily values (u2med, required)
and maximum ones (u2max, optional). If only mean wind values
are available, the function uses a modified version of the 
Guo formula, so that the maximum values are obtained in 
daytime hours.
\end{Description}
%
\begin{Usage}
\begin{verbatim}
hourly_windspeed(climdata)
\end{verbatim}
\end{Usage}
%
\begin{Arguments}
\begin{ldescription}
\item[\code{climdata}] a dataframe with daily wind speed data.
Required columns are Year, Month, Day and u2med. u2max
is an optional data column.
\end{ldescription}
\end{Arguments}
%
\begin{Value}
dataframe with the columns Date, Year, Month, Day, DOY,
Hour and u2 (hourly wind speed, m s-1).
\end{Value}
%
\begin{Author}\relax
Carlos Miranda, \email{carlos.miranda@unavarra.es}
\end{Author}
%
\begin{References}\relax
Guo Z, Chang C, Wang R, 2016. A novel method to downscale daily wind
statistics to hourly wind data for wind erosion modelling. In: Bian F.,
Xie Y. (eds) Geo-Informatics in Resource Management and Sustainable
Ecosystem. GRMSE 2015. Communications in Computer and Information Science,
vol 569. Springer, Berlin, Heidelberg
\end{References}
%
\begin{Examples}
\begin{ExampleCode}

# Generate hourly wind speed for the example dataset
library(magrittr)
library(dplyr)
library(lubridate)
Tudela_Hu2 <- hourly_windspeed(Tudela_DW)

\end{ExampleCode}
\end{Examples}
\inputencoding{utf8}
\HeaderA{moderate\_wind}{Estimation of the daily hours with moderate wind from daily weather data}{moderate.Rul.wind}
%
\begin{Description}\relax
This function estimates the daily hours with wind speed equal or
above than 'moderate breeze' wind (5.5 m s-1 in the Beaufort scale)
from a dataset with daily wind speeds. Hourly wind speeds
from daily values are computed using the formulas proposed
by Guo et al (2016), using mean daily values (u2med, required)
and maximum ones (u2max, optional). If only mean wind values
are available, the function uses a modified version of the 
Guo formula, so that the maximum values are obtained in 
daytime hours.
\end{Description}
%
\begin{Usage}
\begin{verbatim}
moderate_wind(climdata)
\end{verbatim}
\end{Usage}
%
\begin{Arguments}
\begin{ldescription}
\item[\code{climdata}] a dataframe with daily wind speed data.
Required columns are Year, Month, Day and u2med. u2max
is an optional data column.
\end{ldescription}
\end{Arguments}
%
\begin{Value}
dataframe with the columns Date, Year, Month, Day, DOY,
and h\_wind (hours with wind speed equal or above 5.5 m/s).
\end{Value}
%
\begin{Author}\relax
Carlos Miranda, \email{carlos.miranda@unavarra.es}
\end{Author}
%
\begin{References}\relax
Guo Z, Chang C, Wang R, 2016. A novel method to downscale daily wind
statistics to hourly wind data for wind erosion modelling. In: Bian F.,
Xie Y. (eds) Geo-Informatics in Resource Management and Sustainable
Ecosystem. GRMSE 2015. Communications in Computer and Information Science,
vol 569. Springer, Berlin, Heidelberg
\end{References}
%
\begin{Examples}
\begin{ExampleCode}

# Estimate daily hours with wind speed above moderate speeds for the example
# dataset
library(magrittr)
library(dplyr)
library(lubridate)
Tudela_Mu2 <- moderate_wind(Tudela_DW)

\end{ExampleCode}
\end{Examples}
\inputencoding{utf8}
\HeaderA{phenology\_sequential}{Prediction of phenological stages using a sequential model}{phenology.Rul.sequential}
%
\begin{Description}\relax
The function predicts phenological phases for a climate series
from daily chill and heat requirements and daily chill and forcing heat 
data. The sequential model used in the function considers that chilling 
and heat have independent effects. It consists of an accumulation of chill 
up to the plant requirement, followed by heat up to forcing requirement, 
with no overlap between both phases. The function is independent of the
method used to calculate chill and forcing heat, so that chill can
be supplied as chill hours, chill units or chill portions (recommended, 
particularly for warm climates or in climate change studies), forcing heat 
accumulation can be supplied either as GDD or GDH. The function allows
predicting several stages (or the same for different cultivars), by supplying
a dataframe in which each row contains chill and heat requirements for a
phenological stage.
\end{Description}
%
\begin{Usage}
\begin{verbatim}
phenology_sequential(GDH_day, Reqs, Start_chill)
\end{verbatim}
\end{Usage}
%
\begin{Arguments}
\begin{ldescription}
\item[\code{GDH\_day}] a dataframe with daily chilling and forcing accumulation.
It must contain the columns Year, Month, Day, DOY, Chill, GD.

\item[\code{Reqs}] a dataframe in which each row contains the chilling and forcing
heat requirements of one phenological stage. It must contain the columns 
Creq (for chilling) and Freq (for forcing heat).

\item[\code{Start\_chill}] parameter indicating the day of the year when chill
accumulation is supposed to start.
\end{ldescription}
\end{Arguments}
%
\begin{Value}
dataframe with the predicted dates of chilling requirement fulfillment
and date of occurrence of each phenological stage defined in Reqs. Columns are
Creq and Freq (chilling and forcing heat requirements for the phenological stage),
Season, Creq\_Year and Creq\_DOY (year and day of the year in which chill 
requirements are fulfilled), Freq\_Year and Freq\_DOY (year and day of the year
of occurrence the phenological stage).
\end{Value}
%
\begin{Author}\relax
Carlos Miranda, \email{carlos.miranda@unavarra.es}
\end{Author}
%
\begin{Examples}
\begin{ExampleCode}
data(Tudela_DW)
data(Bigtop_reqs)
library(magrittr)
library(dplyr)
library(lubridate)
# Select the first two seasons in Tudela_DW example dataset
Tudela_Sel <- Tudela_DW %>% filter (Tudela_DW$Year<=2002)
# Generate hourly temperatures from the example dataset
Tudela_HT <- hourly_temps(Tudela_Sel,42.13132)
# Calculate chill as chill portions, starting on DOY 305
Chill <- chill_portions(Tudela_HT,305)
# Calculate forcing heat as growing degree hours (GDH) with the linear model,
# using base temperature 4.7 C and no upper thresholds
GDH <- GDH_linear(Tudela_HT,4.7,999,999)
# Combine Chill and GDH values in a dataframe with a format compatible with
# the function phenology_sequential
Tudela_CH <- merge(Chill,GDH) %>%
  select(Date, Year, Month, Day, DOY, Chill,GDH) %>%
    arrange(Date) %>%
    rename(GD=GDH)
# Obtain the predicted dates using the example dataset with requirements
Phenology_BT <- phenology_sequential(Tudela_CH, Bigtop_reqs, 305)

\end{ExampleCode}
\end{Examples}
\inputencoding{utf8}
\HeaderA{phenology\_thermal\_time}{Prediction of phenological stages using a thermal time model}{phenology.Rul.thermal.Rul.time}
%
\begin{Description}\relax
The function predicts phenological phases for a climate series
using a certain starting date and forcing heat requirements data.
The thermal time model used in the function considers that
only heat accumulated from a set date to a given sum explain the
date of occurrence of the phenological stage (i.e, it assumes that
dormancy release occurs before that date). The function is independent
of the method used to calculate forcing heat, so that forcing heat can
be supplied either as GDD or GDH. The function allows predicting several
stages (or the same for different cultivars), by supplying
a dataframe in which each row contains the day for starting forcing
and heat requirements for a phenological stage.
\end{Description}
%
\begin{Usage}
\begin{verbatim}
phenology_thermal_time(GD_day, Reqs)
\end{verbatim}
\end{Usage}
%
\begin{Arguments}
\begin{ldescription}
\item[\code{GD\_day}] a dataframe with daily forcing accumulation. It
must contain the columns Year, Month, Day, DOY, GD.

\item[\code{Reqs}] a dataframe in which each row contains the start date
for forcing and the heat requirements of one phenological stage. 
It must contain the columns Dreq (the start date) and Freq (the forcing 
heat requirements).
\end{ldescription}
\end{Arguments}
%
\begin{Value}
dataframe with the predicted dates of occurrence of each phenological 
stage defined in Reqs. Columns are Dreq and Freq (start date and forcing heat 
requirements for the phenological stage), Season, Dreq\_Year and Dreq\_DOY 
(year and day of the year in which forcing begins), Freq\_Year and Freq\_DOY 
(year and day of the year of occurrence the phenological stage).
\end{Value}
%
\begin{Author}\relax
Carlos Miranda, \email{carlos.miranda@unavarra.es}
\end{Author}
%
\begin{Examples}
\begin{ExampleCode}

# Calculate GDD values from a climate dataset with daily temperature data,
# using a base temperature of 0 C and format it to be compatible with 
# phenology_thermal_time
library(magrittr)
library(dplyr)
library(lubridate)
Tudela_GDD <- GDD_linear(Tudela_DW,0) %>% rename(GD=GDD)
# Create a dataframe with start dates and forcing requirements for
# bloom and veraison in the GFV model for 'Chardonnay' (Parker et al, 
# 2013, Agric Forest Meteorol 180:249-264) in the format required for 
# the function
Dreq <- c(60,60) 
Freq <- c(1217,2547)
Chardonnay_reqs <- as.data.frame(cbind(Dreq,Freq))
# Obtain the predicted dates 
Phenology_Chardonnay <- phenology_thermal_time(Tudela_GDD,Chardonnay_reqs)

\end{ExampleCode}
\end{Examples}
\inputencoding{utf8}
\HeaderA{pollination\_weather}{Evaluation of weather conditions for pollination on a daily series}{pollination.Rul.weather}
%
\begin{Description}\relax
This function estimates the number of days with conditions
favorable, unfavorable and moderately favorable for insect
pollination of fruit trees during the flowering period using
daily weather data.
\end{Description}
%
\begin{Usage}
\begin{verbatim}
pollination_weather(climdata, fendata, lat)
\end{verbatim}
\end{Usage}
%
\begin{Arguments}
\begin{ldescription}
\item[\code{climdata}] a dataframe with daily maximum and minimum temperatures,
wind speed and precipitation. Required columns are Year, Month, Day,
Tmax, Tmin, u2med (daily mean wind speed) and Prec (precipitation).
u2max (daily maximum wind speed) is optional.

\item[\code{fendata}] a dataframe with julian day of the beginning (sbloom)
and end (ebloom) of the flowering season. Must contain the columns
Year, sbloom and ebloom in that order.

\item[\code{lat}] the latitude of the site, in decimal degrees, used to estimate 
hourly temperatures.
\end{ldescription}
\end{Arguments}
%
\begin{Details}\relax
Days are classified considering the classification proposed by
Williams and Sims (1977), by accounting the number of favorable
hours for pollination within a day. One hour is considered favorable
if the temperature is above 12.5 C, the speed of the wind below
4.5 m s-1 and no rainfall occurs (Williams and Sims, 1977; Ramirez and
Davenport, 2013). Hourly wind speeds from daily values are computed
using the formulas proposed by Guo et al (2016), using mean daily
values (u2med, required) and maximum ones (u2max, optional). If 
only mean wind values are available, the function uses a modified
version of the Guo formula, so that the maximum values are obtained in 
daytime hours. No hourly downscaling of rainfall is performed, the 
function allow daily rainfall below 2.0 mm when estimating if a day
is favorable for pollination or not.
\end{Details}
%
\begin{Value}
a data frame with the columns Year, Sbloom (bloom start, DOY)
, Ebloom (end of bloom, DOY), Bloom\_length (in days), Fav\_d (number of
favorable days), Modfav\_d (number of moderately favorable days) and
Unfav\_d (number of unfavorable days).
\end{Value}
%
\begin{Author}\relax
Carlos Miranda, \email{carlos.miranda@unavarra.es}
\end{Author}
%
\begin{References}\relax
Guo Z, Chang C, Wang R, 2016. A novel method to downscale daily wind
statistics to hourly wind data for wind erosion modelling. In: Bian F.,
Xie Y. (eds) Geo-Informatics in Resource Management and Sustainable
Ecosystem. GRMSE 2015. Communications in Computer and Information Science,
vol 569. Springer, Berlin, Heidelberg

Ramirez F and Davenport TL, 2013. Apple pollination: A review. Scientia
Horticulturae 162, 188-203.

Williams RR, Sims FP, 1977. The importance of weather and variability
in flowering time when deciding pollination schemes for Cox's Orange
Pippin. Experimental Horticulture 29, 15-26.
\end{References}
%
\begin{Examples}
\begin{ExampleCode}

# Estimate weather conditions during blooming season using the example
# datasets included in the package
library(magrittr)
library(dplyr)
library(lubridate)
Bloom_BT <- Dates_BT %>% select(Year, sbloom, ebloom) %>% filter(Dates_BT$Year<=2002)
Pol_weather_BT <- pollination_weather(Tudela_DW,Bloom_BT,42.13132)

\end{ExampleCode}
\end{Examples}
\inputencoding{utf8}
\HeaderA{russet}{Estimation of the russet risk for apple and pear fruits}{russet}
%
\begin{Description}\relax
This function assesses the risk of russet in fruit skins. The
risk is defined by the number of hours with the relative humidity (RH)
above a threshold during a given period. For reference, in 'Conference' 
pear the risk is defined by the number of hours with RH> 75\% from 
12 to 30 days after full bloom (Alegre, 2013). In 'Golden' apple, 
the risk is defined by the number of hours with RH> 55\% from 
30 to 34 days after full bloom (Barcelo-Vidal et al., 2013). 
The function requires hourly temperatures and humidity, 
if only daily data is available, the function hourly\_RH can be 
used to estimate them.
\end{Description}
%
\begin{Usage}
\begin{verbatim}
russet(climdata, fendata, RH_crit, init_d, end_d)
\end{verbatim}
\end{Usage}
%
\begin{Arguments}
\begin{ldescription}
\item[\code{climdata}] a dataframe with hourly temperature and RH
data. Required columns are Date, Year, Month, Day, DOY (julian day),
Hour and RH.

\item[\code{fendata}] a dataframe with julian day of occurrence of the full
bloom (F2) phenological stage.
Must contain the columns Year and Fday in that order.

\item[\code{RH\_crit}] the relative humidity threshold

\item[\code{init\_d}] the initial date (as days after full bloom) of the sensitive period

\item[\code{end\_d}] the end date (as days after full bloom) of the sensitive period
\end{ldescription}
\end{Arguments}
%
\begin{Value}
data frame with the number of risk hours (Russet\_hours)
in the sensitive period for each year in the series.
\end{Value}
%
\begin{Author}\relax
Carlos Miranda, \email{carlos.miranda@unavarra.es}
\end{Author}
%
\begin{References}\relax
Alegre S. 2013. Tecnicas de cultivo. In. VII Foro INIA "adaptacion a 
cambio climatico en la produccion fruticola de hueso y pepita". Madrid, 
Spain, pp 1-18
Barcelo-Vidal C, Bonany J, Martin-Fernandez JA and Carbo J. 2013. 
Modelling of weather parameters to predict russet on 'Golden Delicious'
apple. J. Hort. Sci. Biotech. 88: 624-630.
\end{References}
%
\begin{Examples}
\begin{ExampleCode}

# Select the appropiate columns from the example dataset
# Dates_BT and rename column names to make the file compatible
# with the function
library(magrittr)
library(dplyr)
library(lubridate)
Bloom <- Dates_BT %>%
   select(Year, sbloom) %>%
   rename(Fday=sbloom) %>%
   filter(Year==2003)
# Obtain estimated hourly RH from the example dataset Tudela_DW
Weather <- Tudela_DW %>%
        filter (Tudela_DW$Year==2003)
RH_h <- hourly_RH(Weather, 42.13132)
# Estimate the number of russet-inducing days for a RH>55\% 
# between 30 to 34 days after full bloom for each season
Russet_Risk <-russet(RH_h,Bloom,55,30,34)

\end{ExampleCode}
\end{Examples}
\inputencoding{utf8}
\HeaderA{solar\_times}{Estimation of the sunrise and sunset hour}{solar.Rul.times}
%
\begin{Description}\relax
This function estimates the sunrise and sunset hour
for a location, characterized by latitude, and the
day of the year (DOY). The function uses the equations by
Spencer (1971) and Almorox et al. (2005).
\end{Description}
%
\begin{Usage}
\begin{verbatim}
solar_times(latitude, DOY)
\end{verbatim}
\end{Usage}
%
\begin{Arguments}
\begin{ldescription}
\item[\code{latitude}] the latitude of the site, in decimal degrees.

\item[\code{DOY}] numeric value or vector specifying the
day of the year for which calculations should be done.
\end{ldescription}
\end{Arguments}
%
\begin{Value}
list with Sunrise and Sunset times and Daylength.
\end{Value}
%
\begin{Note}\relax
Code adapted from the function \code{\LinkA{daylength}{daylength}}, 
of the \Rhref{https://CRAN.R-project.org/package=chillR}{chillR} Package
\end{Note}
%
\begin{References}\relax
Almorox J, Hontoria C and Benito M, 2005. Statistical validation of
daylength definitions for estimation of global solar radiation in Toledo,
Spain. Energy Conversion and Management 46(9-10), 1465-1471

Luedeling E, 2018. chillR: Statistical Methods for Phenology Analysis in 
Temperate Fruit Trees. R package version 0.70.12. \url{https://CRAN.R-project.org/package=chillR}

Spencer JW, 1971. Fourier series representation of the position of the Sun.
Search 2(5), 172.
\end{References}
%
\begin{Examples}
\begin{ExampleCode}

# Create a vector with 365 days in sequence and calculate sunrise and
# sunset hours for that year in a site placed a 45.5 N
Days <- seq(1:365)
Sunrise_Sunset <- solar_times(41.5,Days)

\end{ExampleCode}
\end{Examples}
\inputencoding{utf8}
\HeaderA{spring\_frost}{Calculates the risk of spring frosts for a climate series}{spring.Rul.frost}
%
\begin{Description}\relax
The function evaluates the number of early and spring frosts and
the expected frost damage on each season within a climate data 
series. Frost damage is assumed to be multiplicative, directly
related to the minimum temperature and unrelated to the duration
of the frost. The function is an enhanced version of the 
Damage Estimator Excel application program (DEST.xls) created by 
de Melo-Abreu and Snyder and bundled with FAO Environment and
Natural Resources Series 10 manual (Snyder and de Melo-Abreu, 2005). 
The function compares daily minimum temperature (Tmin) with the 
critical temperatures (Tcrit) for that day. Daily Tcrit are 
linearly interpolated from a user-provided dataframe with the day 
of occurrence of the stages on each season and a vector of critical 
temperatures (the lethal temperatures for 10\% (LT\_10) and 90\% (LT\_90)
of the organs) for each phenological stage. The main difference of 
spring\_frost with DEST.xls is that the latter uses the same dates of 
phenological occurrence for all the years evaluated (up to 50 years of
data), while spring\_frost is able to use the expected dates of occurrence 
for each year from historical records or estimations produced by the 
functions phenology\_thermal\_time or phenology\_sequential, included in 
this package. There is no limit for the number of years evaluated.
\end{Description}
%
\begin{Usage}
\begin{verbatim}
spring_frost(tempdata, fendata, tcrit, lastday = 181)
\end{verbatim}
\end{Usage}
%
\begin{Arguments}
\begin{ldescription}
\item[\code{tempdata}] a dataframe with daily minimum temperatures for each
year in a series. Must contain the columns Year, julian day of year (DOY) and
the minimum daily temperature (Tmin).

\item[\code{fendata}] a dataframe with julian day of occurrence of the phenological
stages that can be produced by phenology\_thermal\_time or phenology\_sequential 
functions. Must contain the columns Year, and Pheno\_date.

\item[\code{tcrit}] a dataframe with columns LT\_10 and LT\_90,  critical temperatures 
for all the  phenological stages indicated in fendata.

\item[\code{lastday}] the last day (day of the year) to evaluate. By default, 
lastday = 181 (June 30th).
\end{ldescription}
\end{Arguments}
%
\begin{Details}\relax
The last day in the year for the evaluation can be defined by the user, 
and it is set by default set at DOY 181, to avoid computing autumn 
and winter frosts.

The function currently works only with phenological dates occurring
within the same year.
\end{Details}
%
\begin{Value}
a list with two data frames. The df Days\_frost has the columns Year, DOY, 
Tmin, Tcrit and Day\_Frost (indicates a day of frost with a 1 if Tmin<=Tcrit). 
The df Damage\_frosts indicates the total number of frost days and the expected 
damage (as \% of organs) for every year in the series. It has the columns 
Year, Frost\_d, Damage.
\end{Value}
%
\begin{Author}\relax
Carlos Miranda, \email{carlos.miranda@unavarra.es}
\end{Author}
%
\begin{References}\relax
Snyder RL, de Melo-Abreu JP. 2005. Frost Protection: fundamentals, practice and 
economics (2 volumes). FAO Environment and Natural Resources Service Series, 
No. 10 - FAO, Rome.
\end{References}
%
\begin{Examples}
\begin{ExampleCode}
# Generate hourly temperatures from the first season from
# the example dataset Tudela_DW
library(magrittr)
library(dplyr)
library(lubridate)
Tudela_Sel <- Tudela_DW %>% filter (Tudela_DW$Year<=2001)
Tudela_HT <- hourly_temps(Tudela_Sel,42.13132)
# Calculate chill as chill portions, starting on DOY 305
Chill <- chill_portions(Tudela_HT,305)
# Calculate forcing heat as growing degree hours (GDH) with the linear model,
# using base temperature 4.7 C and no upper thresholds
GDH <- GDH_linear(Tudela_HT,4.7,999,999)
# Combine Chill and GDH values in a dataframe with a format compatible with
# the function phenology_sequential
Tudela_CH <- merge(Chill,GDH) %>%
   select(Date, Year, Month, Day, DOY, Chill,GDH) %>%
   arrange(Date) %>%
   rename(GD=GDH)
# Obtain the predicted dates using the example dataset with the requirements 
# indicated in the Bigtop_reqs example dataset and create a dataframe with a
# format compatible with the function spring_frost
Phenology_BT <- phenology_sequential(Tudela_CH, Bigtop_reqs, 305) %>% 
   select(Freq_Year,Freq_DOY) %>%
   rename(Year=Freq_Year,Pheno_date=Freq_DOY)
# Create a dataframe with daily minimum temperatures with the 
# format required by spring_frost
Tmin_Tudela <- Tudela_Sel %>% 
  mutate(Date=make_date(Year,Month,Day), DOY=yday(Date)) %>%
  select(Year, DOY, Tmin) 
# Predict the number and accumulated damage of the spring frosts using the
# critical values contained in the example dataset Tcrits_peach and extract
# the dataframe with the total results for each year
Frost_BT <- spring_frost(Tmin_Tudela, Phenology_BT, Tcrits_peach, 181)
Frost_results <- as.data.frame(Frost_BT[['Damage_frosts']]) 

\end{ExampleCode}
\end{Examples}
\inputencoding{utf8}
\HeaderA{sunburn}{Evaluation of weather conditions for sunburn in apple fruit surface}{sunburn}
%
\begin{Description}\relax
This function estimates the number of days in which apple fruit
surface temperature (FST) exceeds the thresholds indicated by
Rackso and Schrader (2012) for two types of sunburn damages.
\end{Description}
%
\begin{Usage}
\begin{verbatim}
sunburn(climdata, first_d, last_d)
\end{verbatim}
\end{Usage}
%
\begin{Arguments}
\begin{ldescription}
\item[\code{climdata}] a dataframe with daily maximum and minimum temperatures.
Must contain the columns Year, Month, Day, Tmax, Tmin.

\item[\code{first\_d}] Numeric, it is the first date, indicated as day of the year
(DOY), in the assessment.

\item[\code{last\_d}] a vector with the last(s) dates for the assessment (as DOY).
Examples could be harvest dates for several cultivars.
\end{ldescription}
\end{Arguments}
%
\begin{Details}\relax
Sunburn necrosis (SN), the most severe type of sunburn, with a dark
brown or black necrotic spot on the exposed fruit surface is considered
to appear when FST reaches 52ºC. Sunburn browning (SB) is the most 
prevalent type of sunburn on attached sun-exposed apples (acclimated to
high light). The threshold temperature for SB is set in 46ºC, and 
corresponds to the most sensitive apple cultivars (like Cameo or 
Honeycrisp). 

FST is estimated from daily maximum air temperature using the expression
proposed by Schrader et al (2003).
\end{Details}
%
\begin{Value}
data frame with the number of days within the assessed period(s).
Contains the columns Year, Harvest (values from last\_d), SB\_browning
and SB\_necrosis.
\end{Value}
%
\begin{Author}\relax
Carlos Miranda, \email{carlos.miranda@unavarra.es}
\end{Author}
%
\begin{References}\relax
Rackso J and Schrader LE, 2012. Sunburn of apple fruit: Historical
background, recent advances and future perspectives. Critical Reviews
in Plant Sciences 31, 455-504.

Schrader L, Zhang J and Sun J, 2003. Environmental stresses that cause
sunburn of apple. Acta Horticulturae 618, 397-405.
\end{References}
%
\begin{Examples}
\begin{ExampleCode}

# Create one vector with start date (i.e. hand thinning) and a vector 
# with harvest dates to test sunburn risk for several cultivars using.
library(magrittr)
library(dplyr)
library(lubridate)
Thinning_d <- 135
Harvest_d <- c(225,245,260)
Sunburn_risk <- sunburn(Tudela_DW,Thinning_d, Harvest_d)

\end{ExampleCode}
\end{Examples}
\inputencoding{utf8}
\HeaderA{Tcrits\_peach}{Critical frost temperatures for peach flower buds}{Tcrits.Rul.peach}
%
\begin{Description}\relax
Critical frost damage temperatures for peach flower buds for the phenological
stages between 'bud swelling' (B, 51 in Baggliolini and BBCH scales, 
respectively) and 'ovary surrounded by dying sepal crown' (I, 72). For use in 
combination with the example datasets Tudela\_DW and Bigtop\_reqs.
\end{Description}
%
\begin{Usage}
\begin{verbatim}
data("Tcrits_peach")
\end{verbatim}
\end{Usage}
%
\begin{Format}
A data frame with 7 observations on the following 2 variables.
\begin{description}

\item[\code{LT\_10}] a numeric vector, frost temperature causing 10\% kill
\item[\code{LT\_90}] a numeric vector, frost temperature causing 90\% kill

\end{description}

\end{Format}
%
\begin{Details}\relax
The 10\% kill and 90\% kill imply that 30 minutes at the indicated temperature
is expected to cause 10\% and 90\% kill of the flower buds during the phenological
stage. The dataset contains the critical temperatures for the same stages in the
example dataset Bigtop\_reqs. 
\end{Details}
%
\begin{Source}\relax
Miranda C, Santesteban LG, Royo JB. 2005. Variability in the relationship between
frost temperature and injury level for some cultivated Prunus species. HortScience
40:357-361.
\end{Source}
\inputencoding{utf8}
\HeaderA{Tudela\_DW}{Daily weather data from Tudela, Spain}{Tudela.Rul.DW}
%
\begin{Description}\relax
Daily weather data (2000-2010) from Tudela, Spain, recorded at the Tudela-Montes de Cierzo
automatic weather station by Gobierno de Navarra.
\end{Description}
%
\begin{Format}
A data frame with 4018 observations on the following 11 variables.
\begin{description}

\item[\code{Year}] a numeric vector, the observation year
\item[\code{Month}] a numeric vector, the observation month
\item[\code{Day}] a numeric vector, the observation day
\item[\code{Tmax}] a numeric vector, daily maximum temperature in Celsius
\item[\code{Tmin}] a numeric vector, daily minimum temperature in Celsius
\item[\code{RHmax}] a numeric vector, daily maximum relative humidity in \%
\item[\code{RHmin}] a numeric vector, daily minimum relative humidity in \%
\item[\code{Prec}] a numeric vector, daily rainfall in mm
\item[\code{u2med}] a numeric vector, daily mean wind speed in m s-1
\item[\code{u2max}] a numeric vector, daily maximum wind speed in m s-1
\item[\code{Rad}] a numeric vector, daily solar radiation in MJ m-2 day-1

\end{description}

\end{Format}
%
\begin{Source}\relax
http://meteo.navarra.es/estaciones/descargardatos\_estacion.cfm?IDEstacion=36
\end{Source}
\inputencoding{utf8}
\HeaderA{wind\_scab}{Estimation of the risk for wind scab on fruit skin}{wind.Rul.scab}
%
\begin{Description}\relax
The function estimates the risk of wind-induced abrasion injuries
(wind scab) on fruit skin during the sensitive periods of the 
species. This function estimates as risky the daily hours with 
'moderate breeze' wind (equal or above 5.5 m s-1 in the Beaufort 
scale) or stronger, estimated from a dataset with daily wind 
speeds.
\end{Description}
%
\begin{Usage}
\begin{verbatim}
wind_scab(climdata, fendata)
\end{verbatim}
\end{Usage}
%
\begin{Arguments}
\begin{ldescription}
\item[\code{climdata}] a dataframe with daily wind speed data.
Required columns are Year, Month, Day, u2med. u2max is
optional.

\item[\code{fendata}] a dataframe which can contain early fruit growth and
harvest dates. Start of initial growth (Start\_ing) and end of initial growth
(End\_ing) dates are used to assess for early wind-induced abrasion risk and
harvest dates (Harvest) are used for late (pre-harvest) abrasion risks.
Must contain the column Year, and can contain either Start\_ing and End\_ing
or Harvest columns, or the three of them.
\end{ldescription}
\end{Arguments}
%
\begin{Details}\relax
Hourly wind speeds from daily values are computed using the 
formulas proposed by (Guo et al, 2016), using mean daily 
values (u2med, required) and maximum ones (u2max, optional). 
If only mean wind values are available, the function uses a
modified version of the Guo formula, so that the maximum 
values are obtained in daytime hours.

Sensitive periods for wind scab in plums or nectarines
correspond to the early stages of fruit growth, usually the
first three weeks after full bloom (Michailides et al 1992, 
Michailides and Morgan 1992), mainly due to persistent leaf
brushing on fruit skin between the stages '8-mm fruit' and 
'20-mm fruit'. A second sensitive period for cherries, plums, 
peaches and nectarines is pre-harvest (30 days prior to that), 
due to persistent friction against branches. The function 
allows to set both periods or only one of them.
\end{Details}
%
\begin{Value}
data frame with the columns Year, Day\_s, Day\_e, WA\_efg (accumulated
hours with u2>5.5 m s-1 on early fruit growth stage), Day\_h, WA\_bh
(accumulated hours with u2>5.5 m s-1 on the month before harvest).
\end{Value}
%
\begin{Author}\relax
Carlos Miranda, \email{carlos.miranda@unavarra.es}
\end{Author}
%
\begin{References}\relax
Guo Z, Chang C, Wang R, 2016. A novel method to downscale daily wind
statistics to hourly wind data for wind erosion modelling. In: Bian F.,
Xie Y. (eds) Geo-Informatics in Resource Management and Sustainable
Ecosystem. GRMSE 2015. Communications in Computer and Information Science,
vol 569. Springer, Berlin, Heidelberg

Michailides TJ, Morgan DP, Ramirez HT and Giacolini EL, 1992. Determination
of the period when prunes are prone to development of russet scab and
elucidation of the mechanism by which captan controls russet scab.
California dried plum board research reports 1992, 157-175.

Michailides TJ and Morgan DP, 1992. Development of wind scab and
predisposition of french prune fruits to preharvest and postharvest
fungal decay by wind scab and russet scab.California dried plum board
research reports 1992, 149-156.
\end{References}
%
\begin{Examples}
\begin{ExampleCode}

# Select the appropiate columns from the example Dates_BT dataset
# and estimate wind scab risk for Big Top nectarine in Tudela using
# the example weather dataset Tudela_DW
library(magrittr)
library(dplyr)
library(lubridate)
Bloom <- Dates_BT %>%
   select(Year, sbloom) %>%
   rename(Fday=sbloom) %>%
   filter(Year==2003)

Growth_BT <- Dates_BT %>% select(Year, Start_ing, End_ing, Harvest) %>% 
   filter(Year==2003)
Weather <- Tudela_DW %>%
   filter (Tudela_DW$Year==2003)
WindRisk_BT <- wind_scab(Weather, Growth_BT)

\end{ExampleCode}
\end{Examples}
\printindex{}
\end{document}
